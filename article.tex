\documentclass[twoside,twocolumn]{article}

\usepackage{blindtext} 
\usepackage{graphicx}
\usepackage[sc]{mathpazo} 
\usepackage[T1]{fontenc} 
\linespread{1.05} 
\usepackage{microtype} 


\usepackage[english,spanish]{babel} 


\usepackage[hmarginratio=1:1,top=32mm,columnsep=20pt]{geometry} 
\usepackage[hang, small,labelfont=bf,up,textfont=it,up]{caption} 
\usepackage{booktabs} 


\usepackage{lettrine} 


\usepackage{enumitem} 
\setlist[itemize]{noitemsep} 


\usepackage{abstract} 
\renewcommand{\abstractnamefont}{\normalfont\bfseries} 
\renewcommand{\abstracttextfont}{\normalfont\small\itshape} 


\usepackage{titlesec} 
\renewcommand\thesection{\Roman{section}} % 
\renewcommand\thesubsection{\roman{subsection}} 
\titleformat{\section}[block]{\large\scshape\centering}{\thesection.}{1em}{} 
\titleformat{\subsection}[block]{\large}{\thesubsection.}{1em}{} 


\usepackage{fancyhdr} 
\pagestyle{fancy} 
\fancyhead{} 
\fancyfoot{} 
\fancyhead[C]{Comparative Business Analytics vs Business Intelligence \today} 
\fancyfoot[RO,LE]{\thepage} 


\usepackage{titling} 

%----------------------------------------------------------------------------------------
%	TILULOS
%----------------------------------------------------------------------------------------


\setlength{\droptitle}{-4\baselineskip} 

\pretitle{\begin{center}\Huge\bfseries} 
\posttitle{\end{center}} 
\title{Comparative Business Analytics vs Business Intelligence} 
\author{
	 Valdivia Guzman, Alejandra Maria\\
	\and
	Pazos Alarcón, Christian Joshua\\
	\and
	Condori Quispe, Yhónn Joel\\
}
\date{\today} 
\renewcommand{\maketitlehookd}{
\selectlanguage{spanish} 
\begin{abstract}
\noindent 
Lorem ipsum dolor sit amet, consectetur adipiscing elit. Morbi vulputate tempus molestie. 
\end{abstract}
\selectlanguage{english} 
\begin{abstract}
\noindent 
Lorem ipsum dolor sit amet, consectetur adipiscing elit. Morbi vulputate tempus molestie. 
\end{abstract}
}

%----------------------------------------------------------------------------------------

\begin{document}

% Print the title
\maketitle

%----------------------------------------------------------------------------------------
%	INTRODUCCION
%----------------------------------------------------------------------------------------

\section{Introducción}

\lettrine[nindent=0em,lines=3]{L}orem ipsum dolor sit amet, consectetur adipiscing elit.
Morbi vulputate tempus molestie.

%----------------------------------------------------------------------------------------
%	DESARROLLO
%----------------------------------------------------------------------------------------

\section{Desarrollo}

\subsection{Inteligencia de Negocios (BI)}

En 1958, Hans Peter Luhn definió por primera vez la Inteligencia de Negocios como la “habilidad para
aprehender las interrelaciones de hechos en tal forma que guíe la acción hacia el objetivo deseado”.
Este autor plantea que la Inteligencia de Negocios no es solamente un producto, sino una herramienta
que utiliza diferentes tecnologías y en ellas asocia y combina métodos efectivos con determinados
productos, para organizar conjuntos de datos, cuyo uso e interpretación es relevante para mejorar las
utilidades y desempeño de un negocio, además también plantea que tal herramienta permite construir
y aplicar mecanismos capaces de acelerar ciertas acciones y disposiciones sobre el funcionamiento de
los negocios, así como la sistematización de la información clave para la toma de decisiones acertadas\cite{bimurillo2013}.

\begin{center}
	\includegraphics[width=7cm]{./images/business-intelligence}
	Inputs to Business Intelligence Systems\cite{negash2008business}.
\end{center}

\subsection{Componentes de Inteligencia de Negocios}

Convertir los datos dispersos de la organización en información que pueda ser útil para la Inteligencia
Negocios, la toma de decisiones precisas y la provisión de herramientas necesarias para el análisis de datos\cite{bimurillo2013}.

Principales componentes de inteligencia de negocios como lo es la fuente de datos:
\begin{itemize}	
	
	\item \textbf{Datamart}:  Se puede definir como un almacén de datos departamental que se especializa en
	el almacenamiento de datos de una determinada área de negocio. También se conoce como un subconjunto
	de datos derivados del Data Warehouse que está diseñado para soportar los requisitos analíticos específicos
	de una unidad de negocio concreta\cite{bimurillo2013}.
	
	\item \textbf{Data warehouse}:  Base de datos corporativa destinada a gestionar grandes volúmenes de datos
	procedentes de diversas fuentes o tipos, caracterizada por su capacidad de integrar y depurar datos de una o
	varias fuentes diferentes antes de procesarlos de forma que sea posible el análisis desde infinitas perspectivas
	y a gran velocidad de respuesta\cite{bimurillo2013}.
	
\end{itemize} 

\subsection{Infraestructura de Inteligencia de Negocios}

El desarrollo de la inteligencia empresarial se ha centrado principalmente en tres objetivos: La aceleración de la toma
de decisiones ejecutivas, la reducción de costes y la automatización de procesos son objetivos que deben cumplirse, y
para ello es necesario que las bases de datos cumplan los siguientes criterios\cite{bimurillo2013}.

\begin{itemize}	
	
	\item Tener un único punto de acceso inmediato a toda la información independiente de la fuente.
	\item Dar cobertura a todos los procesos empresariales: los análisis multisistema y multiaplicación.
	\item Poseer información de alta calidad (contenido y evaluar los datos en forma flexible).
	\item Ostentar un soporte de alta calidad en la toma de decisiones (gestión operativa y estratégica).
	\item Reducir tiempo y recursos en su implementación(rápido implementar y acceso sencillo y evitar preparación 
	laboriosa de datos heterogéneos).
	\item Poseer información empresarial de alta calidad: datos detallados, recopilados integralmente y presentados de 
	manera multimedia.
	\item Hacer uso de Inteligencia de Negocios y componentes inferiores de data warehousing.
	
\end{itemize} 

%----------------------------------------------------------------------------------------
%	CONCLUSIONES
%----------------------------------------------------------------------------------------

\section{Conclusiones}
\begin{itemize}	
	
	\item Lorem ipsum dolor sit amet, consectetur adipiscing elit. Morbi vulputate tempus molestie.
	\item Lorem ipsum dolor sit amet, consectetur adipiscing elit. Morbi vulputate tempus molestie.

\end{itemize}

%----------------------------------------------------------------------------------------
%	BIBLIOGRAFIA
%----------------------------------------------------------------------------------------

\bibliographystyle{plain} 
\bibliography{references} 
\end{document}