\documentclass[twoside,twocolumn]{article}

\usepackage{blindtext} 
\usepackage{graphicx}
\usepackage[sc]{mathpazo} 
\usepackage[T1]{fontenc} 
\linespread{1.05} 
\usepackage{microtype} 


\usepackage[spanish,english]{babel} 


\usepackage[hmarginratio=1:1,top=32mm,columnsep=20pt]{geometry} 
\usepackage[hang, small,labelfont=bf,up,textfont=it,up]{caption} 
\usepackage{booktabs} 


\usepackage{lettrine} 


\usepackage{enumitem} 
\setlist[itemize]{noitemsep} 


\usepackage{abstract} 
\renewcommand{\abstractnamefont}{\normalfont\bfseries} 
\renewcommand{\abstracttextfont}{\normalfont\small\itshape} 


\usepackage{titlesec} 
\renewcommand\thesection{\Roman{section}} % 
\renewcommand\thesubsection{\roman{subsection}} 
\titleformat{\section}[block]{\large\scshape\centering}{\thesection.}{1em}{} 
\titleformat{\subsection}[block]{\large}{\thesubsection.}{1em}{} 


\usepackage{fancyhdr} 
\pagestyle{fancy} 
\fancyhead{} 
\fancyfoot{} 
\fancyhead[C]{Comparative Business Analytics vs Business Intelligence \today} 
\fancyfoot[RO,LE]{\thepage} 


\usepackage{titling} 

%----------------------------------------------------------------------------------------
%	TILULOS
%----------------------------------------------------------------------------------------


\setlength{\droptitle}{-4\baselineskip} 

\pretitle{\begin{center}\Huge\bfseries} 
\posttitle{\end{center}} 
\title{Comparative Business Analytics vs Business Intelligence} 
\author{
	 Valdivia Guzman, Alejandra Maria\\
	\and
	Pazos Alarcón, Christian Joshua\\
	\and
	Condori Quispe, Yhónn Joel\\
}

\date{\today} 
\renewcommand{\maketitlehookd}{
\selectlanguage{spanish} 
\begin{abstract}
\noindent 
El Business Intelligence y el Business Analytics es un panorama que cambia constantemente en el acelerado mundo
empresarial actual. La riqueza de la información a la que se puede acceder mediante el uso de diversas herramientas
es asombrosa. El problema es que no es fácil encontrar la información que se necesita y darle sentido.\\

Por no hablar de que los términos Business Analytics (BA) y Business Intelligence (BI) se han convertido en sinónimos
con el tiempo. Hay varias diferencias clave entre ambos. Una de las más importantes es que el BA se basa en hacer el
análisis en tiempo real mientras que el BI es más bien un análisis almacenado y posterior.\\

Por lo tanto, el post que se va a escribir hoy tendrá una comparación en profundidad y una idea detallada de los usos y
propósitos entre Business Intelligence y Business Analytics en los negocios.
\end{abstract}
\selectlanguage{english} 
\begin{abstract}
\noindent 
Business Intelligence and Business Analytics is a constantly changing landscape in today’s fast-driven business world.
The wealth of information that can be accessed through the use of various tools is amazing. The problem is that it’s not
easy to find the information you need and to actually make sense of it.\\

Not to mention the terms Business Analytics (BA) and Business Intelligence (BI) have both become relatively synonymous
over time. There are several key differences between these two. One of the most important is that BA relies on doing the
analysis in real-time whereas BI is more of a stored, post-analysis.\\

So the post that is to be penned today will have an in-depth comparison and detailed idea of the uses \& purposes between
Business Intelligence and Business Analytics in business.
\end{abstract}
}

%----------------------------------------------------------------------------------------

\begin{document}

% Print the title
\maketitle

%----------------------------------------------------------------------------------------
%	INTRODUCCION
%----------------------------------------------------------------------------------------

\section{Introduction}

\lettrine[nindent=0em,lines=3]{T}here’s no denying that businesses create an overwhelming amount of data every day. In
order to make smarter business decisions, identify problems, and maintain profitability, it’s crucial to use digital tools and
solutions to turn your data into actionable insights. While many options are available, Business Intelligence and Business
Analytics are the two most commonly used data management solutions that help gather, evaluate, and display key business
information.\\

By utilizing these two digital solutions, organizations gain valuable insights into industry trends and enable a strategic
decision-making approach.\\

Here are a few stats that show the increasing use of Business Intelligence and Business Analytics:

\begin{itemize}	
	
	\item As per Statista, the market size for business intelligence and analytics software applications is expected to increase
	from 14.9 billion U.S. dollars in 2019 to 17.6 billion in 2024.
	\begin{center}
		\includegraphics[width=7cm]{./images/introduction}
	\end{center}
	\item According to a survey by Forbes, 48\% of organizations consider cloud BI either ‘critical’ or ‘very important’
	to their operations.
	\item As per PR News Wire, North America is predicted to have the largest business intelligence and analytics market share
	due to the presence of key industry players in the area.
	
\end{itemize}

Now that we have seen important stats related to BI and BA, let’s understand the concepts in detail and know how and why
Business Analytics and Business Intelligence solutions are important for your organization. Let us also understand the
difference between Business Analytics and Business Intelligence to know which of the solutions is more viable for your
organization.

%----------------------------------------------------------------------------------------
%	DESARROLLO
%----------------------------------------------------------------------------------------

\section{State of Art}

\subsection{Business Intelligence (BI)}

In 1958, Hans Peter Luhn first defined Business Intelligence as the "ability to apprehend the interrelationships
of facts in such a way as to guide action towards the desired objective". This author states that Business Intelligence
is not only a product, but a tool that uses different technologies and in them associates and combines effective methods
with certain products, to organize sets of data, whose use and interpretation is relevant to improve the profits and
performance of a business, and also states that such a tool allows to build and apply mechanisms capable of accelerating
certain actions and provisions on the operation of business, as well as the systematization of key information for making
the right decisions\cite{bimurillo2013}.

\begin{center}
	\includegraphics[width=7cm]{./images/business-intelligence}
	Inputs to Business Intelligence Systems\cite{negash2008business}.
\end{center}

\subsection{Business Intelligence Components}

Convert the organization's scattered data into information that can be useful for business intelligence, accurate decision
making and the provision of the necessary tools for data analysis\cite{bimurillo2013}.

Main components of business intelligence such as data source:
\begin{itemize}	
	
	\item \textbf{Datamart}: It can be defined as a departmental data warehouse that specializes in storing data for a particular
	business area. It is also known as a subset of data derived from the data warehouse that is designed to support the specific
	analytical requirements of a particular business unit\cite{bimurillo2013}.
	
	\item \textbf{Data warehouse}: Corporate database designed to manage large volumes of data from various sources or
	types, characterized by its ability to integrate and clean data from one or more different sources before processing them
	in a way that allows analysis from infinite perspectives and at high response speed\cite{bimurillo2013}.
	
\end{itemize} 

\subsection{Business Intelligence Infrastructure}

The development of business intelligence has focused mainly on three objectives: Acceleration of executive decision
making, cost reduction and process automation are objectives that must be met, and for this to happen, databases
need to meet the following criteria\cite{bimurillo2013}.

\begin{itemize}	
	
	\item Have a single point of immediate access to all information regardless of the source.
	\item Covering all business processes: multi-system and multi-application analysis.
	\item Possess high quality information (content and evaluate data in a flexible way).
	\item To provide high quality support in decision making (operational and strategic management).
	\item Reduce time and resources in its implementation (fast implementation and easy access and avoid laborious
	preparation of heterogeneous data).
	\item Possess high quality business information: detailed data, comprehensively compiled and presented in a multimedia manner.
	\item Make use of Business Intelligence and lower-level data warehousing components.
	
\end{itemize} 

\subsection{Business Analytics (BA)}

Business analytics is about getting useful information from business data to promote the efficiency of an Enterprise and
generate more business values, and the target of business analytics is to get insight from data and support making
fact-based decisions. More specifically, business analytics can be viewed as "a broad umbrella entailing many problems
and solutions, such as demand forecasting and conditioning, resource capacity planning, workforce planning, salesforce
modeling and optimization, revenue forecasting, customer/product analytics, and enterprise recommender systems".
From the perspective of disciplines, business analytics is a branch of management science, which can be seen as an
application of operation research, and a combination of knowledge of signal processing, computer science, and statistics\cite{JIEM3030}.

\subsection{Relation and difference between BA \& BI}

Business analytics and business intelligence are two frequently used terms. However, the relations and differences between
the two are not well-understood, and opinions towards business analytics and business intelligence vary. Chae and Olson
consider business analytics and business intelligence as similar terms, both reflect the use of analytical capabilities for
decision support, whereas Wixom, Yen, and Relich view business analytics as a process and business intelligence as insights
and business intelligence is obtained from business analytics. Gorman and Klimberg think business analytics as an extension
of business intelligence by incorporating advanced statistical and operation research techniques. Varshney and Mojsilovic
share the same idea that "from the managerial perspective, business analytics is an outgrowth of what is known as business
intelligence"\cite{JIEM3030}.

\subsection{Components}

\begin{itemize}	
	
	\item \textbf{Data Aggregation}: Data is collected to one single, central location from where sorting can begin. Inaccurate
	and incomplete data is removed, and only usable data is left behind. Even duplicate data is checked for and removed
	completely. This data is collected from various sources\cite{Ibmr}.
	
	\item \textbf{Data Mining}: To further go deep into the data, Data Mining is the next step to look for unknown patterns
	and trends. For this, one must mine through a huge amount of data by creating mining models. Various statistics
	models are used. One of those models is classification demographics and other such parameters are used for sorting data\cite{Ibmr}.
	
	\item \textbf{Association \& Sequence Identification}: These components are a pattern of consumer behavior. In the association
	part of the behavior, consumers buy products that are associated with each other like toothpaste and toothbrush or shampoo
	and conditioner.\\
	This form of analytics components makes it easier to understand what the consumer is going to buy next and understand their
	buying patterns and behavior\cite{Ibmr}.
	
	\item \textbf{Text Mining}: What consumer types or comments in blogs and other social media comments or their interaction
	with customer service call centers are a part of the text mining component. This data is crucial in improving customer
	service, it can help in the development of new products based on the data collected and help monitor competition and the
	developments they are making\cite{Ibmr}.
	
	\item \textbf{Forecasting}: A famous saying goes that history repeats itself, and this saying is quite true when it comes to the
	forecasting components. It has been observed that consumers resort to a certain behavior, specific seasons or a period.
	This repetitive behavior can be observed and planned for by forecasting\cite{Ibmr}.
	
	\item \textbf{Predictive Analytics}: Helps know when there is going to be a failure or wear and tear in equipment if it has been
	subjected to harsh conditions or has been used for a certain amount of time. It also helps classify customers in detail and
	make decisions based on future trend\cite{Ibmr}.
	
	\item \textbf{Optimization}: They can anticipate surges in demand and step production to maintain supply. They can
	competitively price their products when there is supposed to be a peak or shortage. Businesses can also create
	sales, offers, and discounts based on business analytics\cite{Ibmr}.
	
	\item \textbf{Data Visualization}: Data visualization is one of the most effective ways of presenting data, and business analytics
	are quite helpful. This visual form of data helps companies make reports and sets new goals. The visual format is a lot easier
	to explore, model and analyze\cite{Ibmr}.
	
\end{itemize}

\subsection{Phases}

\begin{itemize}	

	\item \textbf{Descriptive Analytics}: Involves gathering, organizing, and describing the characteristics of the data being studied.
	Traditionally, this is known as "reporting." It is useful in describing what has happened, but it doesn't reveal why something
	happened or what results might happen in the future. Sales and revenue reports are examples of descriptive analytics\cite{Hughson}.

	\item \textbf{Predictive Analytics}: It is concerned with predicting the future by using data from the past. Patterns and
	associations are established among certain variables. The likelihood of an event taking place is then predicted based on
	those patterns and associations. An example is credit card fraud detection. By analyzing commonalities in previous fraudulent
	transactions, credit card companies can detect irregularities and halt suspicious transactions before they are completed\cite{Hughson}.

	\item \textbf{Prescriptive Analytics}: It anticipates what event will happen, when it will happen, and - most importantly - why
	it will happen. The third phase of business analytics is concerned with suggesting a decision or providing options for a course
	of action - much like a doctor would prescribe a specific medicine to treat an ailment. Oil and gas companies use prescriptive
	analytics to decide where to drill, optimize resource extraction, and minimize the impact the extraction process has on
	the environment\cite{Hughson}.

\end{itemize}

\subsection{Architecture}

\begin{center}
	\includegraphics[width=7cm]{./images/architecture}
\end{center}



%----------------------------------------------------------------------------------------
%	CONCLUSIONES
%----------------------------------------------------------------------------------------

\section{Conclusions}
\begin{itemize}	
	
	\item Business analytics is the process of using historical and current data to predict future outcomes, to improve decision-making
	across an organization.
	\item Advantages of an analytics program are that it monitors progress, increases efficiency, and keeps the organization updated with
	real-time information so they can remain competitive. Improper data science business analytics includes a lack of employee access to
	information, takes a long time to work, and low-quality data can deliver poor results.
	\item BI focuses primarily on descriptive analytics while BA focuses on predictive analytics. Both are useful to ensure better
	decision-making, but BA is more critical.
	\item To optimize a business analytics program for growth, use metrics to monitor project management, sales management, and financial management.
	

\end{itemize}

%----------------------------------------------------------------------------------------
%	BIBLIOGRAFIA
%----------------------------------------------------------------------------------------

\bibliographystyle{plain} 
\bibliography{references} 
\end{document}