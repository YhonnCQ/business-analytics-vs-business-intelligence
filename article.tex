\documentclass[twoside,twocolumn]{article}

\usepackage{blindtext} 
\usepackage{graphicx}
\usepackage[sc]{mathpazo} 
\usepackage[T1]{fontenc} 
\linespread{1.05} 
\usepackage{microtype} 


\usepackage[spanish,english]{babel} 


\usepackage[hmarginratio=1:1,top=32mm,columnsep=20pt]{geometry} 
\usepackage[hang, small,labelfont=bf,up,textfont=it,up]{caption} 
\usepackage{booktabs} 


\usepackage{lettrine} 


\usepackage{enumitem} 
\setlist[itemize]{noitemsep} 


\usepackage{abstract} 
\renewcommand{\abstractnamefont}{\normalfont\bfseries} 
\renewcommand{\abstracttextfont}{\normalfont\small\itshape} 


\usepackage{titlesec} 
\renewcommand\thesection{\Roman{section}} % 
\renewcommand\thesubsection{\roman{subsection}} 
\titleformat{\section}[block]{\large\scshape\centering}{\thesection.}{1em}{} 
\titleformat{\subsection}[block]{\large}{\thesubsection.}{1em}{} 


\usepackage{fancyhdr} 
\pagestyle{fancy} 
\fancyhead{} 
\fancyfoot{} 
\fancyhead[C]{Comparative Business Analytics vs Business Intelligence \today} 
\fancyfoot[RO,LE]{\thepage} 


\usepackage{titling} 

%----------------------------------------------------------------------------------------
%	TILULOS
%----------------------------------------------------------------------------------------


\setlength{\droptitle}{-4\baselineskip} 

\pretitle{\begin{center}\Huge\bfseries} 
\posttitle{\end{center}} 
\title{Comparative Business Analytics vs Business Intelligence} 
\author{
	 Valdivia Guzman, Alejandra Maria\\
	\and
	Pazos Alarcón, Christian Joshua\\
	\and
	Condori Quispe, Yhónn Joel\\
}

\date{\today} 
\renewcommand{\maketitlehookd}{
\selectlanguage{spanish} 
\begin{abstract}
\noindent 
Lorem ipsum dolor sit amet, consectetur adipiscing elit. Morbi vulputate tempus molestie. 
\end{abstract}
\selectlanguage{english} 
\begin{abstract}
\noindent 
Lorem ipsum dolor sit amet, consectetur adipiscing elit. Morbi vulputate tempus molestie. 
\end{abstract}
}

%----------------------------------------------------------------------------------------

\begin{document}

% Print the title
\maketitle

%----------------------------------------------------------------------------------------
%	INTRODUCCION
%----------------------------------------------------------------------------------------

\section{Introduction}

\lettrine[nindent=0em,lines=3]{L}orem ipsum dolor sit amet, consectetur adipiscing elit.
Morbi vulputate tempus molestie.

%----------------------------------------------------------------------------------------
%	DESARROLLO
%----------------------------------------------------------------------------------------

\section{State of Art}

\subsection{Business Intelligence (BI)}

In 1958, Hans Peter Luhn first defined Business Intelligence as the "ability to apprehend the interrelationships
of facts in such a way as to guide action towards the desired objective". This author states that Business Intelligence
is not only a product, but a tool that uses different technologies and in them associates and combines effective methods
with certain products, to organize sets of data, whose use and interpretation is relevant to improve the profits and
performance of a business, and also states that such a tool allows to build and apply mechanisms capable of accelerating
certain actions and provisions on the operation of business, as well as the systematization of key information for making
the right decisions\cite{bimurillo2013}.

\begin{center}
	\includegraphics[width=7cm]{./images/business-intelligence}
	Inputs to Business Intelligence Systems\cite{negash2008business}.
\end{center}

\subsection{Business Intelligence Components}

Convert the organization's scattered data into information that can be useful for business intelligence, accurate decision
making and the provision of the necessary tools for data analysis\cite{bimurillo2013}.

Main components of business intelligence such as data source:
\begin{itemize}	
	
	\item \textbf{Datamart}: It can be defined as a departmental data warehouse that specializes in storing data for a particular
	business area. It is also known as a subset of data derived from the data warehouse that is designed to support the specific
	analytical requirements of a particular business unit.\cite{bimurillo2013}.
	
	\item \textbf{Data warehouse}: Corporate database designed to manage large volumes of data from various sources or
	types, characterized by its ability to integrate and clean data from one or more different sources before processing them
	in a way that allows analysis from infinite perspectives and at high response speed.\cite{bimurillo2013}.
	
\end{itemize} 

\subsection{Business Intelligence Infrastructure}

The development of business intelligence has focused mainly on three objectives: Acceleration of executive decision
making, cost reduction and process automation are objectives that must be met, and for this to happen, databases
need to meet the following criteria.\cite{bimurillo2013}.

\begin{itemize}	
	
	\item Have a single point of immediate access to all information regardless of the source.
	\item Covering all business processes: multi-system and multi-application analysis.
	\item Possess high quality information (content and evaluate data in a flexible way).
	\item To provide high quality support in decision making (operational and strategic management).
	\item Reduce time and resources in its implementation (fast implementation and easy access and avoid laborious
	preparation of heterogeneous data).
	\item Possess high quality business information: detailed data, comprehensively compiled and presented in a multimedia manner.
	\item Make use of Business Intelligence and lower-level data warehousing components.
	
\end{itemize} 

%----------------------------------------------------------------------------------------
%	CONCLUSIONES
%----------------------------------------------------------------------------------------

\section{Conclusions}
\begin{itemize}	
	
	\item Lorem ipsum dolor sit amet, consectetur adipiscing elit. Morbi vulputate tempus molestie.
	\item Lorem ipsum dolor sit amet, consectetur adipiscing elit. Morbi vulputate tempus molestie.

\end{itemize}

%----------------------------------------------------------------------------------------
%	BIBLIOGRAFIA
%----------------------------------------------------------------------------------------

\selectlanguage{english} 
\bibliographystyle{plain} 
\bibliography{references} 
\end{document}